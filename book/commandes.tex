% commandes communes entre book et tutorials

\makeatletter
% commande utilisée pour faire une référence croisée entre tutoriels
\newcommand{\iflabelexists}[3]{\@ifundefined{r@#1}{#3}{#2}}
\makeatother
\usepackage{placeins}

\usepackage{pgfplots}
  \pgfplotsset{compat=newest}
  %% the following commands are sometimes needed
  \usetikzlibrary{plotmarks}
  \usepackage{grffile}
  \usepackage{amsmath}
  
 % package pour l'insertion d'un espace intelligent
\usepackage{xspace}
% tilde in math mode
\renewcommand{\t}[1]{\ensuremath{\mathop{\tilde{#1}}\nolimits}}
\newcommand{\h}[1]{\ensuremath{\mathop{\hat{#1}}\nolimits}}
\newcommand{\lab}{$L^*a^*b^*$\xspace}
\newcommand{\argyb}{$(a,rg,yb)$\xspace}
\newcommand{\targyb}{$(\tilde{a},\tilde{rg},\tilde{yb})$\xspace} % tilde
\newcommand{\cargyb}{$(\h{a},\h{rg},\h{yb})$\xspace} % espace chapeau
% TIKZ and diagrams
\usepackage{pgf,tikz}
\usetikzlibrary{arrows,shapes,matrix,positioning}
% styles de block pour diagrammes
\tikzset{decision/.style={diamond, draw, fill=blue!20, text width=1.5cm, text badly centered, inner sep=0pt, minimum width=3.5cm}}
\tikzset{block/.style={rectangle, draw, fill=blue!20, text width=3cm, text badly centered, rounded corners,
minimum width=3cm}}
\tikzset{dligne/.style={draw, latex-latex}}
\tikzset{ligne/.style={draw, -latex}}

\tikzset{title/.style={font=\fontsize{6}{6}\color{black!50}\ttfamily}}
\tikzset{typetag/.style={rectangle, draw=black!50, font=\scriptsize\ttfamily, anchor=west}}

